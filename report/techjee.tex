\section{Techniques JEE employées}

Pour développer le site de e-commerce, nous avons donc utilisé du Java, les 
langages web (HTML/CSS/Javascript), le JSP, les EL et le JSTL. \\

Le Java nous a permis de faire entièrement la partie back-end, ce langage 
constitue donc une majeure partie du projet. \\

Le JSP permet d'intégrer facilement du code java dans les sources web et 
permet d'éviter de faire les écritures sur l'affichage web dans les servlets 
(PrinterWriter), ce qui est déprécié. De plus, nous utilisons les EL afin 
d'avoir une syntaxe plus simple et de pouvoir utiliser les Javabean plus 
facilement. Pour compléter ces deux langages, nous avont aussi intégré du JSTL. 
Il s'agit d'une importation de bibliothèque qui permet d'utiliser plus de 
fonctionnalités dans le JSP. \\

Pour gérer la connexion des utilisateurs et de garder des données durant la 
navigatation, nous avons utilisé les techniques d'utilisation de sessions. 
Cette gestion fait intervenir toutes les techniques précédentes. Il faut 
seulement préciser un détail de scope de session et travailler sur cette 
session dans le code.
