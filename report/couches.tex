\section{Modèle 3-tiers}
En plus du modèle MVC de base, nous avons choisis de suivre un modèle \emph{3 tiers} plus adapté aux sites internet.
Ce modèle permet d'obtenir un site bien structuré et modulaire.

Ce design est composé de 3 couches,
La couche présentation, métier et donnée.

\subsection{La couche Présentation}
Cette couche est le front-end de notre application, elle correspond au code html générer et vue par l'utilisateur, elle touche tout ce qui est navigation dans le site, ergonomie et effet visuel.
Cette couche est donc générer par nos jsp
%TODO more bullshit

\subsection{La couche Métier}
Le couche métier regroupe toute la logique de notre application, c'est ici que nous retrouverons notre MVC2.

Ce dernier consiste à séparer trois 
entités : le Model, la View et le Controller. \\

Le concept de base veut que le Controller se charge des modifications dans le 
Model et d'actualiser la View. Le rôle du Model n'est que de stocker et de 
fournir les procédés de modification de données au Controller. La View est ce 
que voit les utilisateurs. Elle affiche l'interface graphique en fonction de ce 
qu'a demandé le Controller et en fonction des données du Model. \\

Dans le cadre des technologies J2EE, il est souhaité que le Controller soit 
implémenté par une Servlet unique qui sert d'aiguilleur vers les différentes 
pages du site qui sont des View écrites en JSP. Il existe d'autres servlets 
dans notre projet mais qui ne servent seulement à exécuter les requêtes 
déléguées par le Controller.\\

Le Model est représenté par des classes JavaBeans. Une classe JavaBean est une 
classe Java suivant des directives strictes. Il ne doit pas avoir de 
constructeur et ne doit contenir que les attributs, les getter et les setter. 
Les Javabeans permettent de faciliter l'utilisation de la base de données. \\

De manière générale, un navigateur commence par envoyer une requête au 
Controller qui va rediriger l'utilisateur vers la View correspondante et si 
besoin, utiliser le Model pour compléter le contenu de la View qui est renvoyée 
en guise de réponse au navigateur.


\subsection{La couche Donnée}
DAO + Derby
