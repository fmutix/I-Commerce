\section{Structure du projet}

\subsection{Répertoires}

Le projet se divise en deux grands répertoires principaux : le répertoire pour 
les sources java et le répertoire pour les sources web.

\begin{description}
	\item[Web pages] \hfill \\
		css \\
		fonts \\
		img \\
		js \\
		jsp
	\item[Src] \hfill \\
		controller \\
		bean \\
		src
\end{description}

\subsection{web.xml}

\lstinputlisting{../web/WEB-INF/web.xml}

\subsection{Navigation des pages web}

L'ensemble des pages de navigation du site comporte un portail, une page 
d'inscription, une page de connexion, une page listant tous les produits servant 
de vitrine pour les utilisateurs non connectés et un panier. \\

L'index est la page principale une fois l'utilisateur connecté. Elle permet de 
navigeur dans les produits du sites et de faire des recherches filtrées par 
type ou catégorie. \\

D'autres fichiers jsp existants sont utilisés dans les pages de navigation en 
tant que partie de page html à inclure car elles permettent de compléter des 
contenus en fonction de la session utilisateur ou alors d'avoir des structures 
présentent dans toutes les pages à include. Ceci permet une meilleur modularité 
au niveau du développement.
