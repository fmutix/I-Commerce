\section{Structure du projet}

\subsection{Répertoires}

Le projet se divise en deux grands répertoires principaux : le répertoire pour 
les sources java et le répertoire pour les sources web.

\begin{description}
	\item[Web pages] \hfill \\
		css \\
		fonts \\
		img \\
		js \\
		jsp
	\item[Src] \hfill \\
		controller \\
		bean \\
		src
\end{description}

\subsection{web.xml}

\lstinputlisting{../web/WEB-INF/web.xml}

\subsection{Navigation des pages web}

L'ensemble des pages de navigation du site comporte un portail, une page 
d'inscription, une page de connexion, une page listant tous les produits servant 
de vitrine pour les utilisateurs non connectés et un panier. \\

L'index est la page principale une fois l'utilisateur connecté. Elle permet de 
navigeur dans les produits du sites et de faire des recherches filtrées par 
type ou catégorie. \\

D'autres fichiers jsp existants sont utilisés dans les pages de navigation en 
tant que partie de page html à inclure car elles permettent de compléter des 
contenus en fonction de la session utilisateur ou alors d'avoir des structures 
présentent dans toutes les pages à include. Ceci permet une meilleur modularité 
au niveau du développement.

\subsection{Architecture MVC2}

Pour programmer un site bien structuré et modulable, nous avons suivi le 
design pattern MVC adapté pour un projet JEE. Ce dernier consite à séparer trois 
entités : le Model, la View et le Controller. \\

Le concept de base veut que le Controller se charge des modifications dans le 
Model et d'actualiser la View. Le rôle du Model n'est que de stocker et de 
fournir les procédés de modification de données au Controller. La View est ce 
que voit les utilisateurs. Elle affiche l'interface graphique en fonction de ce 
qu'a demandé le Controller et en fonction des données du Model. \\

Dans le cadre des technologies JEE, il est souhaité que le Controller soit 
implémenté par une Servlet unique qui sert d'aiguilleur vers les différentes 
pages du site qui sont des View écrites en JSP. Il existe d'autres servlets 
dans notre projet mais qui ne servent seulement à exécuter les requêtes 
déléguées par le Controller.\\

Le Model est représenté par des classes JavaBeans. Une classe JavaBean est une 
classe Java suivant des directives strictes. Il ne doit pas avoir de 
constructeur et ne doit contenir que les attributs, les getter et les setter. 
Les Javabeans permettent de faciliter l'utilisation de la base de données. \\

De manière générale, un navigateur commence par envoyer une requête au 
Controller qui va rediriger l'utilisateur vers la View correspondante et si 
besoin, utiliser le Model pour compléter le contenu de la View qui est renvoyée 
en guise de réponse au navigateur.
